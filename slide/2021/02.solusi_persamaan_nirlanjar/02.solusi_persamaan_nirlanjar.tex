\documentclass[pdflatex,compress]{beamer}

%\usetheme[dark,framenumber,totalframenumber]{ElektroITK}
\usetheme[darktitle,framenumber,totalframenumber]{ElektroITK}

\usepackage{graphicx}

\title{METODE NUMERIK}
\subtitle{Solusi Persamaan Nirlanjar}

\author{Tim Dosen Pengampu}

\begin{document}
	
\maketitle

\section{Rumusan Masalah}

\begin{frame}
	\frametitle{Rumusan Masalah}
	\begin{itemize}
		\item \textbf{Persoalan:} Temukan nilai $ x $ yang memenuhi persamaan
		\[f(x) = 0\]
		yaitu nilai $ x = s $ sedemikian sehingga $ f(s) = 0 $.
		\item Nilai $ x = s $ disebut \textbf{akar} persamaan $ f(x) = 0 $
	\end{itemize}
\end{frame}

\begin{frame}
	\frametitle{Contoh persoalan dalam bidang elektronika}
	Suatu arus osilasi dalam rangkaian listrik diberikan oleh
	
	\[ I = 10e^{-t}\sin(2 \pi t) \]
	
	yang dalam hal ini $ t $ dalam detik. Tentukan semua nilai $ t $
	sedemikan sehingga $ I = 2 $ ampere.\\
	Persoalan ini adalah mencari nilai $ t $ sedemikian sehingga:

	\[ 10e^{-t} \sin(2 \pi t) – 2 = 0 \]
\end{frame}

\end{document}
