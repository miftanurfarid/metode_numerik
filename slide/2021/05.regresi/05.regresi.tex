\documentclass[pdflatex,compress,mathserif]{beamer}

%\usetheme[dark,framenumber,totalframenumber]{ElektroITK}
\usetheme[darktitle,framenumber,totalframenumber]{ElektroITK}

\usepackage[utf8]{inputenc}
\usepackage[T1]{fontenc}
\usepackage{lmodern}
\usepackage[bahasai]{babel}
\usepackage{amsmath}
\usepackage{amsfonts}
\usepackage{amssymb}
\usepackage{graphicx}
\usepackage{multicol}
\usepackage{lipsum}

\newcommand*{\Scale}[2][4]{\scalebox{#1}{$#2$}}%

\title{METODE NUMERIK}
\subtitle{Regresi}

\author{Tim Dosen Pengampu}

\begin{document}

\maketitle

\section{Sub-CMPK dan Bahan Kajian}

\begin{frame}
	\frametitle{Sub-CMPK dan Bahan Kajian}
	\begin{itemize}
		\item \textbf{Sub-CPMK:} Mahasiswa mampu melakukan regresi numerik
		
	\end{itemize}
\end{frame}

\end{document}
